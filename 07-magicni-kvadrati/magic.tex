\documentclass[a4paper,12pt]{article}
\usepackage[slovene]{babel}
\usepackage[utf8]{inputenc}
\usepackage[T1]{fontenc}
\usepackage{lmodern}
\usepackage{amsthm}

\usepackage{url}
\usepackage{graphicx}
\usepackage{booktabs}
\usepackage{amsmath}
\usepackage{amssymb}
\usepackage{xcolor}

\theoremstyle{definition}
\newtheorem{definicija}{Definicija}


\renewcommand*\proofname{Dokaz Izreka}


\theoremstyle{plain}
\newtheorem{izrek}{Izrek}



\newenvironment{magic}[3]{
   \begin{table}[h]
      \Large
      \centering
      \caption{#1}
      \label{#3}
      \begin{tabular}{|*{#2}{c|}}
         \hline
         }{
      \end{tabular}
   \end{table}
}




% Nastavimo koliko prostora izpustimo pod napisom nad sliko ali tabelo.
\setlength{\belowcaptionskip}{2mm}
% Nastavimo odmik prve vrstice v odstavku na 0mm.
\setlength{\parindent}{0mm}


\newcommand{\m}{\mathcal{M}_2}
\newcommand{\pojem}[1]{\emph{\color{purple}#1}}

%%%%%%%%%%%%%%%%%%%%%%%%%%%%%%%%%%%%%%%%%%%%%%%%%%%%%%%%%%%%%%%%%%%%%%%%
% Prvi sklop nalog
%%%%%%%%%%%%%%%%%%%%%%%%%%%%%%%%%%%%%%%%%%%%%%%%%%%%%%%%%%%%%%%%%%%%%%%%

\title{Magični kvadrati}
% 1. naloga
% Glavo dokumenta popravite tako, da se ne bo izpisal datum.
% To naredite tako, da za ukazom `\title` dodate ukaz `\date{}`.
\date{}

\begin{document}

\maketitle

% 2. naloga
% Na tem mestu vključite sliko `slika.pdf`:
% - naredite okolje `center`, da bo slika postavljena na sredino
% - v okolju `center' uporabite ukaz  `\includegraphics{slika.pdf}`

\begin{center}
   \includegraphics{slika.pdf}
\end{center}

Prirejeno iz virov:

% 3. naloga
% Za izpis spletnih naslovov morate uporabiti ukaz `\url`.
% S tem ukazom lahko izpišete en spletni naslov tako,
% naslov podate kot argument ukaza `\url` v zavitih oklepajih.
% V spodnjem okolju `itemize` ukaz `\url` na obeh spletnih naslovih
% (naslova sta v komentarju, da ne pride do napake pri prevajanju te datoteke).
\begin{itemize}
   \item \url{http://mathworld.wolfram.com/MagicSquare.html}
   \item \url{http://en.wikipedia.org/wiki/Magic_square}
\end{itemize}

% 4. naloga
% Na tem mestu uporabite ukaz, ki izpiše kazalo vsebine.
% Poiščite ga na spletu: po angleško se kazalu vsebine reče "table of contents".

\tableofcontents

%%%%%%%%%%%%%%%%%%%%%%%%%%%%%%%%%%%%%%%%%%%%%%%%%%%%%%%%%%%%%%%%%%%%%%%%
% Konec prvega sklopa nalog
%%%%%%%%%%%%%%%%%%%%%%%%%%%%%%%%%%%%%%%%%%%%%%%%%%%%%%%%%%%%%%%%%%%%%%%%

% Z ukazom `\newpage` lahko povzročite prehod na novo stran.
\newpage

\section{Uvod}

\begin{definicija}
   \pojem{Magični kvadrat} reda $n$ je nabor $n^2$ različnih števil,
   ki so razvrščena v kvadratno tabelo tako, da vedno dobimo enako vsoto,
   če seštejemo vsa števila poljubne vrstice, vsa števila poljubnega
   stolpca ali vsa števila v katerikoli od glavnih diagonal.
\end{definicija}

Primer magičnega kvadrata reda 3 je prikazan v tabeli \ref{table:mag3}.



\begin{magic}{Magični kvadrat reda 3.}{3}{table:mag3}
   8 & 1 & 6 \\\hline
   3 & 5 & 7 \\\hline
   4 & 9 & 2 \\\hline
\end{magic}


\begin{definicija}
   Magični kvadrat reda $n$ je \pojem{normalen}, če v njem nastopajo števila
   \begin{equation}
      % oznaka: eq:numbers
      1, 2, 3, \ldots, n^2-1, n^2.
   \end{equation}
\end{definicija}

Magični kvadrat v tabeli \ref{table:mag3} je normalen.
To je tudi najmanjši netrivialen normalen magični kvadrat.
Poleg normalnih magičnih kvadratov so zanimivi tudi magični kvadrati praštevil.

%%%%%%%%%%%%%%%%%%%%%%%%%%%%%%%%%%%%%%%%%%%%%%%%%%%%%%%%%%%%%%%%%%%%%%%%

\section{Zgodovina}

\subsection{Kvadrat ">Lo Shu"<}

Kitajska literatura iz časa vsaj 2800 let pred našim štetjem govori o legendi
\pojem{Lo Shu} -- ">zvitek reke Lo"<. V antični Kitajski je prišlo do
silne poplave. Ljudje so skušali rečnemu bogu narasle reke Lo ponuditi daritev,
da bi pomirili njegovo jezo. Iz vode se je prikazala želva z zanimivim vzorcem
na oklepu: v tabeli velikosti tri krat tri so bila predstavljena števila, tako
da je bila vsota števil v katerikoli vrstici, kateremkoli stolpcu in na obeh
glavnih diagonalah enaka: 15. To število je tudi enako številu dni v 24 ciklih
kitajskega sončnega leta. Ta vzorec so na določen način uporabljali upravljalci
reke.




\begin{magic}{Kvadrat Lo Shu.}{3}{table:loshu}
   4 & 9 & 2 \\\hline
   3 & 5 & 7 \\\hline
   8 & 1 & 6 \\\hline
\end{magic}


%%%%%%%%%%%%%%%%%%%%%%%%%%%%%%%%%%%%%%%%%%%%%%%%%%%%%%%%%%%%%%%%%%%%%%%%

\subsection{Kulturna pomembnost}

Magični kvadrati so fascinirali človeštvo skozi vso zgodovino. Najdemo jih
v številnih kulturah, npr.\ v Egiptu in Indiji, vklesane v kamen ali
kovino, uporabljane kot talismane za dolgo življensko dobo in v
izogib boleznim.

\pojem{Kubera-Kolam} je talna poslikava, ki se uporablja v Indiji, in je v
obliki magičnega kvadrata reda 3. Ta je v bistvu enak kot kvadrat
Lo Shu, vendar je vsako število povečano za 19.



\begin{magic}{Kvadrat Kubera-Kolam.}{3}{table:kubera}
   23 & 28 & 21 \\\hline
   22 & 24 & 26 \\\hline
   27 & 20 & 25 \\\hline
\end{magic}


Z magičnimi kvadrati so se ukvarjali tudi najbolj znani matematiki kot na
primer Euler, glej \cite{euler}. % ključ: euler

%%%%%%%%%%%%%%%%%%%%%%%%%%%%%%%%%%%%%%%%%%%%%%%%%%%%%%%%%%%%%%%%%%%%%%%%

\subsection{Zgodnji kvadrati reda 4}

Najzgodnejši znani magični kvadrat reda 4 je bil odkrit na napisu
v Khajurahu v Indiji in v Enciklopediji Bratovščine Čistosti iz enajstega
ali dvanajstega stoletja. Vrh vsega gre celo za ">panmagični kvadrat"<.
V Evropi sta morda najbolj znana naslednja magična kvadrata reda 4:

Magični kvadrat v litografiji Melancholia I (glej sliko \ref{fig:durer}
za izsek s kvadratom) Albrechta Dürerja naj bi bil najzgodnejši magični kvadrat
v evropski umetnosti. Zelo podoben je kvadratu Yang Huija, ki je nastal na Kitajskem
približno 250 let pred Dürerjevim časom. % sklic: fig:durer

% Tu vstavite sliko
% - napis nad sliko: Dürerjev magični kvadrat
% - oznaka: fig:durer
% Dodatni parameter v oglatih oglepajih sliko nastavi velikost
% slike na 150% originalne velikosti.

\begin{figure}[h]
   \centering
   \caption{Dürerjev magični kvadrat}
   \includegraphics[scale=1.5]{durer.png}
   \label{fig:durer}
\end{figure}

Vsoto 34 je mogoče najti pri seštevanju števil v vsaki vrstici, vsakem stolpcu,
na vsaki diagonali, v vsakem od štirih kvadrantov, v sredinskih štirih poljih,
v štirih kotih, v štirih sosedih kotov v smeri urinega kazalca ($3+8+14+9$), v
štirih sosedih kotov v nasprotni smeri urinega kazalca ($2+5+15+12$), v dveh naborih
simetričnih parov ($2+8+9+15$ in $3+5+12+14$), in še na nekaj drugih načinov.
Števili na sredini spodnje vrstici tvorita letnico litografije: 1514.



\begin{magic}{Dürerjev magični kvadrat $4\times 4$.}{4}{table:durer}
   16 &  3 &  2 & 13 \\\hline
   5 & 10 & 11 &  8 \\\hline
   9 &  6 &  7 & 12 \\\hline
   4 & 15 & 14 &  1 \\\hline
\end{magic}





Pasijonska fasada na katedrali Sagrada família v Barceloni
(glej sliko \ref{fig:sagrada} za fotografijo) vsebuje magični kvadrat reda 4.
% sklic: fig:sagrada

% Tu vstavite sliko
% - napis nad sliko: Pasijonska fasada, Sagrada Família
% - oznaka: fig:sagrada
\begin{figure}[h]
   \centering
   \caption{Pasijonska fasada, Sagrada Família}
   \includegraphics{sagrada.png}
   \label{fig:sagrada}
\end{figure}

Vsota števil v vrsticah, stolpcih oziroma na diagonalah je 33 -- Jezusova starost
v času pasijona. Strukturno je kvadrat podoben Dürerjevemu, vendar so števila
v štirih poljih zmanjšana za 1. Posledica je, da sta števili 10 in 14 podvojeni
in zato kvadrat ni normalen.
%
% - napis nad tabelo: Pasijonska fasada, Sagrada Família
% - oznaka: table:sagrada
%    1 & 14 & 14 &  4 \\\hline
%   11 &  7 &  6 &  9 \\\hline
%    8 & 10 & 10 &  5 \\\hline
%   13 &  2 &  3 & 15 \\\hline






\begin{magic}{Pasijonska fasada, Sagrada Família.}{4}{table:sagrada}
   1 & 14 & 14 &  4 \\\hline
   11 &  7 &  6 &  9 \\\hline
    8 & 10 & 10 &  5 \\\hline
   13 &  2 &  3 & 15 \\\hline
\end{magic}



%%%%%%%%%%%%%%%%%%%%%%%%%%%%%%%%%%%%%%%%%%%%%%%%%%%%%%%%%%%%%%%%%%%%%%%%

\section{Osnovne lastnosti}

\begin{definicija}
      Vsoto ene vrstice, enega stolpca ali ene od glavnih diagonal
      v magičnem kvadratu imenujemo \pojem{magična konstanta}.
\end{definicija}

\begin{izrek}
      Magična konstanta normalnega magičnega kvadrata reda $n$
      je enaka
      \begin{equation}
         % oznaka: eq:mc
         \m(n) = \frac{1}{2} n(n^2+1)
         \label{eq:mc}
      \end{equation}
\end{izrek}



\begin{proof}
      V normalnem magičnem kvadratu reda $n$ je vsota vseh nastopajočih
      števil (glej \eqref{eq:mc} na strani \pageref{eq:mc}) enaka
      $1+2+3+\dots+n^2=\sum_{k=1}^{n^2}k=\frac{1}{2}n^2(n^2+1)$. Ker imamo
      v kvadratu $n$ vrstic z enako vsoto, je vsota števil v eni vrstici
      enaka številu $\m(n)$. % sklica: eq:numbers, eq:numbers
\end{proof}

Preprost račun pokaže, da je konstanti \eqref{eq:mc} analogna konstanta
$\m(n;A,D)$ za magični kvadrat, v katerem so nameščena števila
$A$, $A+D$, $A+2D$, \dots, $A+(n^2-1)D$, enaka % sklic: eq:mc
\begin{equation}
   \,(n; A,D) = \frac{1}{2} (2A + D(n^2 -1))
\end{equation}
Kvadratu v tabeli \ref{table:kubera} ustrezata konstanti $A=20$ in $D=1$.

\begin{definicija}
      Če vsako od števil v normalnem magičnem kvadratu reda $n$ odštejemo
      od števila $n^2+1$, dobimo nov magični kvadrat, ki je prvotnemu
      \pojem{komplementaren}.
\end{definicija}

Na primer, magičnemu kvadratu Lo Shu (glej tabelo \ref{table:loshu}) priredimo
komplementarni kvadrat, prikazan v tabeli \ref{table:closhu}.



\begin{magic}{Kvadratu Lo Shu komplementarni kvadrat}{3}{table:closhu}
   6 & 1 & 8 \\\hline
   7 & 5 & 3 \\\hline
   2 & 9 & 4 \\\hline
\end{magic}

Vidimo, da je dobljeni kvadrat moč dobiti iz kvadrata Lo Shu tudi z zasukom za
180 stopinj okrog središča, kvadrat iz tabele \ref{table:closhu} pa je mogoče dobiti
iz kvadrata Lo Shu z zrcaljenjem preko sredinske vodoravne črte.

Število različnih normalnih magičnih kvadratov

\begin{definicija}
      Pravimo, da sta dva magična kvadrata \pojem{različna}, če enega ni mogoče dobiti
      iz drugega s pomočjo zasukov oziroma zrcaljenj.
\end{definicija}

Števila različnih normalnih magičnih kvadratov se nahajajo v tabeli \ref{table:stevila}.


\begin{table}[h]
   \centering
   \caption{Število različnih normalnih magičnih kvadratov}
   \begin{tabular}{lcccccc} \toprule
      \multicolumn{5}{c}{točna vrednost} & približek \\\midrule
      red & 1 & 2 & 3 & 4 & 5 & 6\\
      število kvadratov & 1 & 0 & 1 & 880 & 275305224 & (1,7745 $\pm$ 0,0016)$10^{19}$\\
      \bottomrule
   \end{tabular}
   \label{table:stevila}
\end{table}





Vse normalne magične kvadrate reda 4 je oštevilčil Frénicle de Bessy
leta 1693, glej \cite{bessy}, in jih je moč najti v knjigi \cite{berlekamp}
iz leta 1982. Število normalnih kvadratov reda 5 je izračunal
R. Schroeppel leta 1973 (glej Gardner \cite{gardner}).
Natančno število vseh različnih normalnih magičnih kvadratov reda 6 ni znano.
Avtorja navedenega približka sta Pinn in Wieczerkowski (glej \cite{pinn}), ki
sta za oceno uporabila simulacijo Monte Carlo in metode statistične mehanike.
 % ključi: bessy, berlekamp, gardner, pinn

%%%%%%%%%%%%%%%%%%%%%%%%%%%%%%%%%%%%%%%%%%%%%%%%%%%%%%%%%%%%%%%%%%%%%%%%

\section{Primeri}

V tabelah \ref{table:mag5}, \ref{table:mag6} in \ref{table:mag9} so prikazani
magični kvadrati redov 5, 6 in 9.





\begin{magic}{Magični kvadrat reda 5}{5}{table:mag5}
   17 & 24 &  1 &  8 & 15 \\\hline
   23 &  5 &  7 & 14 & 16 \\\hline
    4 &  6 & 13 & 20 & 22 \\\hline
   10 & 12 & 19 & 21 &  3 \\\hline
   11 & 18 & 25 &  2 &  9 \\\hline
\end{magic}





\begin{magic}{Magični kvadrat reda 6}{5}{table:mag6}
   17 & 24 &  1 &  8 & 15 \\\hline
   23 &  5 &  7 & 14 & 16 \\\hline
    4 &  6 & 13 & 20 & 22 \\\hline
   10 & 12 & 19 & 21 &  3 \\\hline
   11 & 18 & 25 &  2 &  9 \\\hline
\end{magic}





\begin{magic}{Magični kvadrat reda 9}{9}{table:mag9}
   47 & 58 & 69 & 80 &  1 & 12 & 23 & 34 & 45 \\\hline
   57 & 68 & 79 &  9 & 11 & 22 & 33 & 44 & 46 \\\hline
   67 & 78 &  8 & 10 & 21 & 32 & 43 & 54 & 56 \\\hline
   77 &  7 & 18 & 20 & 31 & 42 & 53 & 55 & 66 \\\hline
    6 & 17 & 19 & 30 & 41 & 52 & 63 & 65 & 76 \\\hline
   16 & 27 & 29 & 40 & 51 & 62 & 64 & 75 &  5 \\\hline
   26 & 28 & 39 & 50 & 61 & 72 & 74 &  4 & 15 \\\hline
   36 & 38 & 49 & 60 & 71 & 73 &  3 & 14 & 25 \\\hline
   37 & 48 & 59 & 70 & 81 &  2 & 13 & 24 & 35 \\\hline
\end{magic}


%%%%%%%%%%%%%%%%%%%%%%%%%%%%%%%%%%%%%%%%%%%%%%%%%%%%%%%%%%%%%%%%%%%%%%%%

\newpage

% Tu vstavite bibliografijo

\bibliographystyle{siam}

\bibliography{magic}


%%%%%%%%%%%%%%%%%%%%%%%%%%%%%%%%%%%%%%%%%%%%%%%%%%%%%%%%%%%%%%%%%%%%%%%%

\end{document}
